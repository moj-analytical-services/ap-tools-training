% Options for packages loaded elsewhere
\PassOptionsToPackage{unicode}{hyperref}
\PassOptionsToPackage{hyphens}{url}
%
\documentclass[
]{book}
\title{Analytical Platform and related tools training}
\author{MoJ coding training leads}
\date{2022-04-05}

\usepackage{amsmath,amssymb}
\usepackage{lmodern}
\usepackage{iftex}
\ifPDFTeX
  \usepackage[T1]{fontenc}
  \usepackage[utf8]{inputenc}
  \usepackage{textcomp} % provide euro and other symbols
\else % if luatex or xetex
  \usepackage{unicode-math}
  \defaultfontfeatures{Scale=MatchLowercase}
  \defaultfontfeatures[\rmfamily]{Ligatures=TeX,Scale=1}
\fi
% Use upquote if available, for straight quotes in verbatim environments
\IfFileExists{upquote.sty}{\usepackage{upquote}}{}
\IfFileExists{microtype.sty}{% use microtype if available
  \usepackage[]{microtype}
  \UseMicrotypeSet[protrusion]{basicmath} % disable protrusion for tt fonts
}{}
\makeatletter
\@ifundefined{KOMAClassName}{% if non-KOMA class
  \IfFileExists{parskip.sty}{%
    \usepackage{parskip}
  }{% else
    \setlength{\parindent}{0pt}
    \setlength{\parskip}{6pt plus 2pt minus 1pt}}
}{% if KOMA class
  \KOMAoptions{parskip=half}}
\makeatother
\usepackage{xcolor}
\IfFileExists{xurl.sty}{\usepackage{xurl}}{} % add URL line breaks if available
\IfFileExists{bookmark.sty}{\usepackage{bookmark}}{\usepackage{hyperref}}
\hypersetup{
  pdftitle={Analytical Platform and related tools training},
  pdfauthor={MoJ coding training leads},
  hidelinks,
  pdfcreator={LaTeX via pandoc}}
\urlstyle{same} % disable monospaced font for URLs
\usepackage{longtable,booktabs,array}
\usepackage{calc} % for calculating minipage widths
% Correct order of tables after \paragraph or \subparagraph
\usepackage{etoolbox}
\makeatletter
\patchcmd\longtable{\par}{\if@noskipsec\mbox{}\fi\par}{}{}
\makeatother
% Allow footnotes in longtable head/foot
\IfFileExists{footnotehyper.sty}{\usepackage{footnotehyper}}{\usepackage{footnote}}
\makesavenoteenv{longtable}
\usepackage{graphicx}
\makeatletter
\def\maxwidth{\ifdim\Gin@nat@width>\linewidth\linewidth\else\Gin@nat@width\fi}
\def\maxheight{\ifdim\Gin@nat@height>\textheight\textheight\else\Gin@nat@height\fi}
\makeatother
% Scale images if necessary, so that they will not overflow the page
% margins by default, and it is still possible to overwrite the defaults
% using explicit options in \includegraphics[width, height, ...]{}
\setkeys{Gin}{width=\maxwidth,height=\maxheight,keepaspectratio}
% Set default figure placement to htbp
\makeatletter
\def\fps@figure{htbp}
\makeatother
\setlength{\emergencystretch}{3em} % prevent overfull lines
\providecommand{\tightlist}{%
  \setlength{\itemsep}{0pt}\setlength{\parskip}{0pt}}
\setcounter{secnumdepth}{5}
\usepackage{booktabs}
\ifLuaTeX
  \usepackage{selnolig}  % disable illegal ligatures
\fi
\usepackage[]{natbib}
\bibliographystyle{apalike}

\begin{document}
\maketitle

{
\setcounter{tocdepth}{1}
\tableofcontents
}
The Analytical Platform team is not responsible for the content on this page.

\hypertarget{summary}{%
\chapter{Summary of available resources}\label{summary}}

\hypertarget{getting-set-up-on-the-analytical-platform}{%
\section{Getting set up on the Analytical Platform}\label{getting-set-up-on-the-analytical-platform}}

Please follow the steps in the \href{https://user-guidance.services.alpha.mojanalytics.xyz/get-started.html\#get-started}{Getting Started section of the Analytical Platform User Guide}. To use the Analytical Platform you then need to access the \href{https://controlpanel.services.alpha.mojanalytics.xyz/tools/}{Analytical Platform Control Panel} and from there open the relevant tool (e.g.~RStudio for R).

As code written on the Analytical Platform should be stored in a Git repository on GitHub, complete the steps \href{https://user-guidance.services.alpha.mojanalytics.xyz/github.html\#setup-github-keys-to-access-it-from-r-studio-and-jupyter}{to configure Git and GitHub for the Analytical Platform}.

\hypertarget{r-mentoring}{%
\section{R mentoring}\label{r-mentoring}}

It is recommended that all who are new to R or DASD request an R mentor. The purpose of the scheme is to provide a better on-the-job R learning experience and raise awareness of the preferred DASD ways of working that will for instance enable people to get up to speed more quickly with others' code. The scheme is also open to non-coders who need to use the Analytical Platform to advise them through the learning process, and for those commencing a more complex project involving the use of R. To request an R mentor please complete \href{https://forms.office.com/Pages/ResponsePage.aspx?id=KEeHxuZx_kGp4S6MNndq2PdkMZw8L-FEkMSL2t4Oet1UQVlWMUwzOVZRNUVKSTVZU0pPTUY0MDlZTSQlQCN0PWcu}{this mentee form}. If you could become an R mentor (we have a shortage of mentors) please complete \href{https://forms.office.com/Pages/ResponsePage.aspx?id=KEeHxuZx_kGp4S6MNndq2PdkMZw8L-FEkMSL2t4Oet1UNUw1N1M0NVNQSTNOVkdHOUtMQ1lMT0lHTSQlQCN0PWcu}{this mentor form}. For more information please contact Jessica Dawson.

\hypertarget{internal-r-and-sql-training-group-materials}{%
\section{Internal R (and SQL) Training Group materials}\label{internal-r-and-sql-training-group-materials}}

The live R, SQL and GitHub training sessions are generally run twice or three times each year. However, recordings are available so you can work through yourself. These sessions are run using the MoJ Analytical Platform so they are recommended for those working in DASD (or with the Analytical Platform). You can view the material (in the Github repositories) at:

\begin{itemize}
\tightlist
\item
  \href{https://github.com/moj-analytical-services/IntroRTraining}{Introduction to R}
\item
  \href{https://github.com/moj-analytical-services/ggplotTraining}{R Charting}
\item
  \href{https://github.com/moj-analytical-services/git-training-class}{Introduction to GitHub}
\item
  \href{https://github.com/moj-analytical-services/rmarkdown_training}{R Markdown}
\item
  \href{https://github.com/moj-analytical-services/r-excel-training}{Interfacing Excel with R}
\item
  \href{https://github.com/moj-analytical-services/writing_functions_in_r}{Writing Functions in R}
\item
  \href{https://github.com/moj-analytical-services/rpackage_training}{Developing R packages \& RAP ways of working}
\item
  \href{https://github.com/moj-analytical-services/sql_training}{Introduction to SQL}
\end{itemize}

The recordings of the R and SQL training sessions (the Introduction to GitHub session has not yet been recorded) are stored on MS Stream in the links below:

\begin{itemize}
\tightlist
\item
  \href{https://web.microsoftstream.com/channel/aa3cda5d-99d6-4e9d-ac5e-6548dd55f52a}{R training MS Stream Channel}
\item
  \href{https://web.microsoftstream.com/channel/7cd1cdaf-79cb-4e1e-ab2b-448d8f69f6a1}{SQL training MS Stream Channel}
\end{itemize}

One great way of learning is by teaching. If you would be interested in being part of the R (and SQL) training group, whether booking courses/allocating places, designing training, or presenting, please contact Aidan Mews or Georgina Eaton.

If you have any questions please contact Aidan Mews or Georgina Eaton.

\hypertarget{coffee-and-coding-and-bitesize-sessions}{%
\section{Coffee and Coding and Bitesize sessions}\label{coffee-and-coding-and-bitesize-sessions}}

The internal R/SQL/Github training (see above) is complemented by \href{https://web.microsoftstream.com/channel/f6aa6c5d-e90c-44b7-8ccc-28a318fa0630}{coffee and coding} and \href{https://web.microsoftstream.com/channel/5a6012a2-efd6-4b86-902d-98c864427caa}{bitesize} sessions. The contacts are Katharine Breeze and George Papadopoulos respectively.

\hypertarget{analytical-function-training-opportunities}{%
\section{Analytical Function training opportunities}\label{analytical-function-training-opportunities}}

There are now many technical Analytical Function training opportunities for analysts including about topics not presently covered internally e.g.~an introduction to python. Examples useful for RAP practitioners include \href{https://gss.civilservice.gov.uk/training/best-practice-in-programming-clean-code/}{Best practice in programming -- clean code -- GSS (civilservice.gov.uk)} and an \href{Introduction\%20to\%20unit\%20testing\%20–\%20GSS\%20(civilservice.gov.uk)}{Introduction to unit testing -- GSS (civilservice.gov.uk)}. You can view such opportunities via:

\begin{itemize}
\tightlist
\item
  \href{https://www.gov.uk/guidance/af-learning-curriculum-technical}{A user-friendly list (but not updated since Jan 2021)}
\item
  \href{https://gss.civilservice.gov.uk/training-courses/}{GSS Training Courses (many which are open to all analysts)}
\end{itemize}

\hypertarget{datacamp}{%
\section{Datacamp}\label{datacamp}}

Paid Datacamp licenses are beneficial to cover gaps in current training provision that are not picked up by either internal or GSS/Analytical function training currently e.g.~more advanced R, SQL, and python skills. You can read more about Datacamp \href{https://www.datacamp.com/}{here}. If you would like to have a Datacamp license please get approval from your deputy director and contact Aidan Mews.

\hypertarget{other-assistance}{%
\section{Other assistance}\label{other-assistance}}

Technical help can be requested via the following \href{https://asdslack.slack.com/}{DASD slack channels}:

\begin{itemize}
\tightlist
\item
  intro\_R which provides support to those starting out in the world of R and RAP.
\item
  R which is for beginners and experts alike.
\item
  sql
\item
  python
\end{itemize}

You may also find useful a trello board providing \href{https://trello.com/b/D5pSkqnT/online-analytical-training}{Links to further free online analytical training including in R} and \href{https://rstudio.com/resources/cheatsheets/}{R cheatsheets}.

  \bibliography{book.bib,packages.bib}

\end{document}
